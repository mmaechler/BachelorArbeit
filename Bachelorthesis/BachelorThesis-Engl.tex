%%%--- Template for bachelor thesis at SfS, based on template
%%%--- for Diploma and Master thesis------
%%%------
\documentclass[11pt,a4paper,twoside,openright]{report}
\usepackage[english]{ETHDAsfs}%--> ETHDASA + fancyhdr + ... "umlaute"
%  + sfs-hyper -> hyperref 

\usepackage{pdfpages}%%to include the confirmation of originality (plagiarism
\usepackage{amsbsy}%% for \boldsymbol and \pmb{.}
\usepackage{amssymb}%% calls  amsfonts...
\usepackage{graphicx}%-- für PostScript-Grafiken (besser als  psfig!)
%\usepackage[draft]{graphicx} % grafics shown as boxes --> faster compilation
%
\usepackage[longnamesfirst]{natbib}%was {sfsbib}%- Für  Literatur-Referenzen
%           ^^^^^^^^^^^^^^ 1) "Hampel, Ronchetti, ..,"  2) "Hampel et al"
% Engineers (and other funny people) want to see [1], [2] 
% ---> use 'numbers' : \usepackage[longnamesfirst,number]{natbib}
%
%
\usepackage{texab}%- 'tex Abkürzungen' /u/sfs/tex/tex/latex/texab.sty
        %%- z.B.  \R, \Z, \Q, \Nat für reelle, ganze, rationale, natürl. Zahlen;
        %%-       \N   (Normalvert.)  \W == Wahrscheinlichkeit .....
        %%-  \med, \var, \Cov, \....
        %%-  \abs{x} == |x|   und   \norm{y} ==  || y ||   (aber anständig)
%% NOTE: texab contains many useful definitions and "shortcuts". It is
%% worth to open the file and have a look at them. HOWEVER, some
%% definitions are a bit can lead to conflicts with other packages. You
%% might for example want to comment out the line defininf \IF as an
%% operator when working with the algorithmic package, or to comment out
%% the line defining a command \Cite with working with the Biblatex package  
\usepackage{amsmath}
%\usepackage{mathrsfs}% Raph Smith's Formal Script font --> provides \mathscr
\usepackage[utf8]{inputenc}% <<------- Unicode, *NOT* iso-latin1 !
\usepackage{ae}% A[lmost] E[uropean] Fonts
\usepackage{enumerate}% Fuer selbstdefinierte Nummerierungen
%--------
\usepackage{relsize}%-> \smaller (etc) used here
\usepackage{color} %% to allow coloring in code listings
\usepackage{Sweave}
\usepackage{listings}% Fuer R-code, C-code, ....  and settings for these:

\usepackage{lineno}  %TODO comment this out when not needed anymore

\definecolor{Mygrey}{gray}{0.75}% for linenumbers only!
\definecolor{Cgrey}{gray}{0.4}% for comments
\lstloadlanguages{R}
%%--- first version of "listings of R"-style : ---------------------------
% %% using \smaller here: makes R code listings use a *small* font:
% \lstset{language=R,basicstyle=\smaller[2],commentstyle=\rmfamily\smaller,
%   showstringspaces=false,xleftmargin=4ex,
%   literate={<-}{{$\leftarrow$}}1 {~}{{$\sim$}}1}
% \lstset{escapeinside={(*}{*)}} % for (*\ref{ }*) inside lstlistings (Scode) 
%\newcommand{\lil}[1]{\lstinline|#1|}
%%--- newer version of "listings of R"-style : ---------------------------
\lstset{%% Help, e.g. --> https://en.wikibooks.org/wiki/LaTeX/Source_Code_Listings
language=R,
basicstyle=\ttfamily\scriptsize,%%- \small > \footnotesize > \scriptsize > \tiny
%commentstyle=\ttfamily\color{Cgrey},
commentstyle=\itshape\color{Cgrey},
numbers=left,
numberstyle=\ttfamily\color{Mygrey}\tiny,
stepnumber=1,
numbersep=5pt,
backgroundcolor=\color{white},
showspaces=false,
showstringspaces=false,
showtabs=false,
frame=single,
tabsize=2,
captionpos=b,
breaklines=true,
%breakatwhitespace=false,
keywordstyle={},
morekeywords={},
xleftmargin=4ex, 
literate={<-}{{$\leftarrow$}}1 {~}{{$\sim$}}1}
\lstset{escapeinside={(*}{*)}} % for (*\ref{ }*) inside lstlistings (Scode) 
%%----------------------------------------------------------------------------

%%------- Theoreme ---
\newtheorem{definition}{Definition}[subsection]
\newtheorem{lemma}[definition]{Lemma}
\newtheorem{theorem}[definition]{Theorem}
\newtheorem{Coro}[definition]{Corollary}
\theoremstyle{definition} 
\newtheorem{example}[definition]{Example}
\newtheorem*{note}{Note}
\newtheorem*{remark}{Remark}

\DeclareMathOperator*{\plim}{plim}
\def\MR#1{\href{http://www.ams.org/mathscinet-getitem?mr=#1}{MR#1}}

\newcommand{\Lecture}[3]{\marginpar{#3.#2.#1}}
\newcommand{\Fu}{\mathcal{F}}
\newcommand{\aatop}[2]{\genfrac{}{}{0pt}{}{#1}{#2}}

%\renewcommand{\theequation}{\arabic{equation}}
\numberwithin{equation}{subsection}
%\includeonly{}

%\pscompress %--compress the Epsf .ps files and produce .bb -- WHEN dvips is run
%\psdraft%--- for quick pre-viewing (only shows frames)

%%%%%%%%%%%%%%%%%%%%%%%%%%%%%%%%%%%%%%%%%%%%%%%%%
%%% Path for your figures                      %%%
%%%%%%%%%%%%%%%%%%%%%%%%%%%%%%%%%%%%%%%%%%%%%%%%%
% Set the paths where all figures are taken from:
% \graphicspath{{/u/mueller/MA/Figures1/}{/u/mueller/MA/Figures2/}}

%%%%%%%%%%%%%%%%%%%%%%%%%%%%%%%%%%%%%%%%%%%%%%%%%
%%% Define your own commands here             %%%
%%%%%%%%%%%%%%%%%%%%%%%%%%%%%%%%%%%%%%%%%%%%%%%%%
\newcommand{\Bruch}[2]{{}^{#1}\!\!/\!_{#2}}
\renewcommand{\labelenumi}{\roman{enumi}.)}





\begin{document}
\bibliographystyle{chicago}% ---> Hampel,F., E.Ronchetti,... W.Stahel(1986) ...
 %was \bibliographystyle{sfsbib}\citationstyle{dcu} %OR DEFAULT : \citationstyle{agsm}

\pagenumbering{roman}%- roman numbering for first few pages

%%%%%%%%%%%%%%%%%%%%%%%%%%%%%%%%%%%%%%%%%%%%%%%%%
%%% Title page                                %%%
%%%%%%%%%%%%%%%%%%%%%%%%%%%%%%%%%%%%%%%%%%%%%%%%%
 \period{Winter 2019} %e.g. Winter 2019
 \dasatype{Bachelor Thesis}
 \students{Nicolas Trutmann} %e.g. Patrick M\"uller
 \mainreaderprefix{Advisor:}
 \mainreader{placeholder} %e.g. Prof. Dr. Martin M\"achler
 \alternatereaderprefix{}
 \alternatereader{}
 \issuedate{placeholder} %e.g. September 15th 2019
 \submissiondate{placeholder} %e.g. February 31st 2019
 \title{Comparison of EM-algorithm and MLE using Cholesky decomposition} %title of your work

 \maketitle%- Titelseite wird abgeschlossen
 %%~~~~~~~~~~~~~~~~~~~~~~~~~~~~~~~~~~~~~~~~

%%%%%%%%%%%%%%%%%%%%%%%%%%%%%%%%%%%%%%%%%%%%%%%%%
%%% Insert here acknowledgements and abstract %%%
%%%%%%%%%%%%%%%%%%%%%%%%%%%%%%%%%%%%%%%%%%%%%%%%%
 \newpage
\begin{abstract}
	The intent of this work is to compare The EM algorithm to a MLE approach in the
	case of multivariate normal mixture models using the Cholesky decomposition.
	The EM algorithm is widely used in statistics and is proven to converge, 
	however in pathological cases convergence slows down considerably. 

	methods(not done)

	results(not done)
\end{abstract}

%%%%%%%%%%%%%%%%%%%%%%%%%%%%%%%%%%%%%%%%%%%%%%%%%
%%% Table of contents and list of figures and %%%   
%%% tables (no need to change this usually)   %%%
%%%%%%%%%%%%%%%%%%%%%%%%%%%%%%%%%%%%%%%%%%%%%%%%%
\newpage
\tableofcontents
\newpage
\listoffigures
\newpage
\listoftables
\cleardoublepage

\pagenumbering{arabic}%--- switch back to standard numbering 


%%%%%%%%%%%%%%%%%%%%%%%%%%%%%%%%%%%%%%%%%%%%%%%%%
%%% Your text... Either write here directly,  %%%
%%% or even better: write in separate files   %%%
%%% that you just have to include here.       %%% 
%%%%%%%%%%%%%%%%%%%%%%%%%%%%%%%%%%%%%%%%%%%%%%%%%
\linenumbers  %% TODO remove after not needed anymore
\chapter{Introduction to normal mixture models}% Umlaute work thanks to \inputencoding{..} % in main file

here intro to normal mixtures

explain in scetch EM algo

explain idea to use parameter optimizer instead,
EM has pathological insufficiencies, like 'getting stuck' for many iterations.
we hope we need less iterations, and as concequence less time.
'special' idea: using cholesky decomp.


\section{choice of notation}

describe difference in notation between ceuleux \& govaert and our covariance matrix decomposition.

make clear that the models can not be translated one to one to ldlt model

make nice table(maybe sideways to account for parameter list)


\rotatebox{90}{
	\begin{tabular}{c c c c c}
		\hline
		Model & $\pmb{\Sigma}_k$ C\&G & volume & shape & orientation \\
		\hline

		EII	& $ \alpha \pmb{I} $ & equal & & \\
		VII	& $ \alpha_k \pmb{I} $ 		& variable & & \\
		EEI	& $ \alpha \pmb{\Lambda} $ 	& equal & & \\
		VEI	& $ \alpha_k \pmb{\Lambda} $ & variable && \\
		EVI	& $ \alpha \pmb{\Lambda}_k $ &equal && \\
		VVI	& $ \alpha_k \pmb{\Lambda}_k $ & variable && \\
		EEE	& $ \alpha \pmb{Q \Lambda Q}^\top $ &equal && \\
		EVE	& $ \alpha \pmb{Q \Lambda}_k \pmb{Q}^\top $ &equal && \\
		VEE	& $ \alpha_k \pmb{Q \Lambda Q}^\top $ & variable && \\
		VVE	& $ \alpha_k \pmb{Q \Lambda}_k \pmb{Q}^\top $ &variable && \\
		EEV	& $ \alpha \pmb{Q}_k \pmb{\Lambda} \pmb{Q}_k^\top $ &equal && \\
		VEV	& $ \alpha_k \pmb{Q}_k \pmb{\Lambda} \pmb{Q}_k^\top $ &variable && \\
		EVV	& $ \alpha \pmb{Q}_k \pmb{\Lambda}_k \pmb{Q}_k^\top $ &equal && \\
		VVV	& $ \alpha_k \pmb{Q}_k \pmb{\Lambda}_k \pmb{Q}_k^\top $&variable &&
	\end{tabular}
	}

\chapter{This is chapter 2!}

\section{First section}
\ldots


%%% Local Variables: 
%%% mode: latex
%%% TeX-master: "BachelorThesis-Engl"
%%% End: 

\chapter{Comparing Algorithms}


\section{Time Analysis}

here how much time they take, in p,k and n give approximate O(x) value

\begin{Schunk}
\begin{Sinput}
>     library(norMmix, lib.loc="~/ethz/BA/norMmix.Rcheck/")
>     mainsav <- normalizePath("~/ethz/BA/Rscripts/")
>     savdir <- file.path(mainsav, "2time")
>     filelist <- list.files(savdir, pattern=".rds")
>     filelist <- grep("mcl.rds", filelist, invert=TRUE, value=TRUE)
>     f <- lapply(file.path(savdir,filelist), function(j) readRDS(j)$fit)
>     times <- unlist(lapply(f, function(j) extracttimes(j)[,,1]))
>     dims <- unlist(lapply(f, function(j) attr(extracttimes(j), "p")))
>     size <- unlist(lapply(f, function(j) attr(extracttimes(j), "n")))
>     ddims <- rep(dims, each=80)
>     ssize <- rep(size, each=80)
>     pars <- unlist(lapply(f, npar))
>     r <- lm(log(times) ~ log(pars) + log(ddims) + log(ssize))
>     summary(r)
\end{Sinput}
\begin{Soutput}
Call:
lm(formula = log(times) ~ log(pars) + log(ddims) + log(ssize))

Residuals:
    Min      1Q  Median      3Q     Max 
-3.4428 -0.2986  0.0671  0.4579  2.0936 

Coefficients:
            Estimate Std. Error t value Pr(>|t|)    
(Intercept) -9.74133    0.10598  -91.91   <2e-16 ***
log(pars)    2.75983    0.01181  233.75   <2e-16 ***
log(ddims)  -2.06063    0.02483  -82.99   <2e-16 ***
log(ssize)   0.61301    0.01446   42.38   <2e-16 ***
---
Signif. codes:  0 ‘***’ 0.001 ‘**’ 0.01 ‘*’ 0.05 ‘.’ 0.1 ‘ ’ 1

Residual standard error: 0.6946 on 7196 degrees of freedom
Multiple R-squared:  0.8887,	Adjusted R-squared:  0.8887 
F-statistic: 1.916e+04 on 3 and 7196 DF,  p-value: < 2.2e-16
\end{Soutput}
\end{Schunk}

\begin{figure}[h]
    \centering
\begin{Schunk}
\begin{Sinput}
>     plot(times~pars, log="xy", yaxt="n", xaxt="n")
>     sfsmisc::eaxis(1)
>     sfsmisc::eaxis(2)
\end{Sinput}
\end{Schunk}
\includegraphics{chapter3-figtime}
    \caption{Log-log Plot of System Time against Parameter Length}
    \label{fig:time}
\end{figure}

can see that time is almost one to one proportional to parameter length.

\section{Behaviour in {\tt n}}

% it 1
here show as expected narrower scattering as n increases


\begin{figure}[h]
    \centering
\begin{Schunk}
\begin{Sinput}
>     compplot(s05mw34bic, m0534)
\end{Sinput}
\end{Schunk}
\includegraphics{chapter3-fig5fit}
\end{figure}[h]
\begin{figure}
\begin{Schunk}
\begin{Sinput}
>     compplot(s10mw34bic, m1034)
\end{Sinput}
\end{Schunk}
\includegraphics{chapter3-fig10fit}
\end{figure}


\section{Behaviour in {\tt p}}

% it 1
here show how norMmix is consistently competitive with mclust

\begin{figure}[h]
\begin{Schunk}
\begin{Sinput}
>     plot(MW34)
\end{Sinput}
\end{Schunk}
\includegraphics{chapter3-figMW34}
\end{figure}



\begin{figure}[h]
\begin{Schunk}
\begin{Sinput}
>     compplot(clarabic, mclbic, mclustbic, main="Fit of MW34")
\end{Sinput}
\end{Schunk}
\includegraphics{chapter3-figMW34bic}
\end{figure}

\section{Diffixult Mixtures}

% it 1
here show behaviour in difficult cases

\begin{figure}[h]
\begin{Schunk}
\begin{Sinput}
>     plot(MW214)
\end{Sinput}
\end{Schunk}
\includegraphics{chapter3-figMW214}
\end{figure}


\begin{Schunk}
\begin{Sinput}
>     savdir <- file.path(mainsav, "2init")
>     filenames <- list.files(savdir, pattern=".rds")
>     MW214fn <- grep("MW214", filenames, value="TRUE")
>     mclustfiles <- grep("mcl.rds", MW214fn, value=TRUE)
>     MW214fn <- grep("mcl.rds", MW214fn, value="TRUE", invert=TRUE)
>     claraMW <- grep("clara", MW214fn, value=TRUE)
>     mclMW <- grep("mclVVV", MW214fn, value=TRUE)
>     clarabic <- massbic(claraMW, savdir)
>     mclbic <- massbic(mclMW, savdir)
>     mclustbic <- readRDS(file.path(savdir,mclustfiles[1]))
\end{Sinput}
\end{Schunk}

\begin{figure}[h]
\begin{Schunk}
\begin{Sinput}
>     compplot(clarabic, mclbic, mclustbic, main="Fit of MW214")
\end{Sinput}
\end{Schunk}
\includegraphics{chapter3-figMW214bic}
\end{figure}

\section{Nonnormal Mixtures}


\chapter{Discussion}


one shortcoming is time inefficiency. largely due to implementation.
mclust has 16'000 lines of Fortran code, impossible in the scope of this thesis.

proof of concept??
definitely possible to do model selection using a general optimizer.

strong points:
'randomness' of clara/optim allows 'confidence intervalls' for selected model
flexibility of approach: given an logLik fctn can do mixture fitting w/ arbitrary
models

further study might include: other presumed component distributions, 'high' dimensions


%% and more
% \include{....}
 
%%%%%%%%%%%%%%%%%%%%%%%%%%%%%%%%%%%%%%%%%%%%%%%%%
%%% Bibliography                              %%%
%%%%%%%%%%%%%%%%%%%%%%%%%%%%%%%%%%%%%%%%%%%%%%%%%
\addtocontents{toc}{\vspace{.5\baselineskip}}
% \cleardoublepage
\phantomsection
\addcontentsline{toc}{chapter}{\protect\numberline{}{Bibliography}}

\nocite{sfs19}
\nocite{mvt19}
\nocite{clu19}
\nocite{mix09}
\nocite{mas02}


\bibliography{References}
%% All books from our library (SfS) are already in a BiBTeX file
%% 'Assbib.bib' (included here as well), using
% \bibliography{myReferences,Assbib}
% ---------------------------------- instead of the above

%%%%%%%%%%%%%%%%%%%%%%%%%%%%%%%%%%%%%%%%%%%%%%%%% 
%%% Appendices (if needed, e.g. for R code)   %%%
%%%%%%%%%%%%%%%%%%%%%%%%%%%%%%%%%%%%%%%%%%%%%%%%%
\addtocontents{toc}{\vspace{.5\baselineskip}}
\appendix
\chapter{\Rp Code}


\section{{\tt llnorMmix}}

% it 1
Here {\tt llnorMmix}, since it is the central piece of the package,
and 2time.R as an example of a simulation script.


\begin{Schunk}
\begin{Soutput}
#### the llnorMmix function, calculating log likelihood for a given
#### parameter vector
## Author: Nicolas Trutmann 2019-07-06
## Log-likelihood of parameter vector given data
#
# par:   parameter vector
# tx:    transposed sample matrix
# k:     number of components
# model: assumed distribution model of normal mixture
# trafo: either centered log ratio or logit
llnorMmix <- function(par, tx, k,
                      trafo=c("clr1", "logit"),
                      model=c("EII","VII","EEI","VEI","EVI",
                              "VVI","EEE","VEE","EVV","VVV")
                      ) {
    stopifnot(is.matrix(tx),
              length(k <- as.integer(k)) == 1, k >= 1)
    p <- nrow(tx)
#    x <- t(x) ## then only needed in   (x-mu[,i])^2  i=1..k
    # 2. transform
    model <- match.arg(model)
    trafo <- match.arg(trafo)
    l2pi <- log(2*pi)
    # 3. calc log-lik
    # get w
    w <- if (k==1) 1
         else switch(trafo,
                     "clr1" = clr1inv (par[1:(k-1)]),
                     "logit"= logitinv(par[1:(k-1)]),
                     stop("invalid 'trafo': ", trafo)
         )
    # start of relevant parameters:
    f <- k + p*k # weights -1 + means +1 => start of alpha
    # get mu
    mu <- matrix(par[k:(f-1L)], p,k)
    f1 <- f      # end of alpha if uniform
    f2 <- f+k-1L # end of alpha if var
    f1.1 <- f1 +1L # start of D. if alpha unif.
    f2.1 <- f1 + k # start of D. if alpha variable
    f11 <- f1 + p-1    # end of D. if D. uniform and alpha uniform
    f12 <- f1 +(p-1)*k # end    D. if D.   var   and alpha unif.
    f21 <- f2 + p-1    # end of D. if D. uniform and alpha variable
    f22 <- f2 +(p-1)*k # end of D. if D.   var   and alpha var.
    f11.1 <- f11 +1L # start of L if alpha unif  D unif
    f21.1 <- f21 +1L # start of L if alpha var   D unif
    f12.1 <- f12 +1L # start of L if alpha unif  D var
    f22.1 <- f22 +1L # start of L if alpha var   D var
    f111 <- f11 +   p*(p-1)/2 # end of L if alpha unif  D unif
    f211 <- f21 +   p*(p-1)/2 # end of L if alpha var   D unif
    f121 <- f12 + k*p*(p-1)/2 # end of L if alpha unif  D var
    f221 <- f22 + k*p*(p-1)/2 # end of L if alpha var   D var
    # initialize f(tx_i) i=1..n  vector of density values
    invl <- 0
    # calculate log-lik, see first case for explanation
    switch(model,
    "EII" = {
        alpha <- par[f]
        invalpha <- exp(-alpha)# = 1/exp(alpha)
        for (i in 1:k) {
            rss <- colSums(invalpha*(tx-mu[,i])^2)
            # this is vector of length n=sample size
            # calculates (tx-mu)t * Sigma^-1 * (tx-mu) for diagonal
            # cases.
            invl <- invl+w[i]*exp(-0.5*(p*(alpha+l2pi)+rss))
            # adds likelihood of one component to invl
            # the formula in exp() is the log of likelihood
            # still of length n
        }
    },
    # hereafter differences are difference in dimension in alpha and D.
    # alpha / alpha[i] and D. / D.[,i]
    "VII" = {
        alpha <- par[f:f2]
        for (i in 1:k) {
            rss <- colSums((tx-mu[,i])^2/exp(alpha[i]))
            invl <- invl+w[i]*exp(-0.5*(p*(alpha[i]+l2pi)+rss))
        }
    },
    "EEI" = {
        alpha <- par[f]
        D. <- par[f1.1:f11]
        D. <- c(-sum(D.),D.)
        D. <- D.-sum(D.)/p
        invD <- exp(alpha+D.)
        for (i in 1:k) {
            rss <- colSums((tx-mu[,i])^2/invD)
            invl <- invl+w[i]*exp(-0.5*(p*(alpha+l2pi)+rss))
        }
    },
    "VEI" = {
        alpha <- par[f:f2]
        D. <- par[f2.1:f21]
        D. <- c(-sum(D.), D.)
        D. <- D.-sum(D.)/p
        for (i in 1:k) {
            rss <- colSums((tx-mu[,i])^2/exp(alpha[i]+D.))
            invl <- invl+w[i]*exp(-0.5*(p*(alpha[i]+l2pi)+rss))
        }
    },
    "EVI" = {
        alpha <- par[f]
        D. <- matrix(par[f1.1:f12],p-1,k)
        D. <- apply(D.,2, function(j) c(-sum(j), j))
        D. <- apply(D.,2, function(j) j-sum(j)/p)
        for (i in 1:k) {
            rss <- colSums((tx-mu[,i])^2/exp(alpha+D.[,i]))
            invl <- invl+w[i]*exp(-0.5*(p*(alpha+l2pi)+rss))
        }
    },
    "VVI" = {
        alpha <- par[f:f2]
        D. <- matrix(par[f2.1:f22],p-1,k)
        D. <- apply(D.,2, function(j) c(-sum(j), j))
        D. <- apply(D.,2, function(j) j-sum(j)/p)
        for (i in 1:k) {
            rss <- colSums((tx-mu[,i])^2/exp(alpha[i]+D.[,i]))
            invl <- invl+w[i]*exp(-0.5*(p*(alpha[i]+l2pi)+rss))
        }
    },
    # here start the non-diagonal cases. main difference is the use
    # of backsolve() to calculate tx^t Sigma^-1 tx, works as follows:
    # assume Sigma = L D L^t, then Sigma^-1 = (L^t)^-1 D^-1 L^-1
    # y = L^-1 tx  => tx^t Sigma^-1 tx = y^t D^-1 y
    # y = backsolve(L., tx)
    "EEE" = {
        alpha <- par[f]
        D. <- par[f1.1:f11]
        D. <- c(-sum(D.), D.)
        D. <- D.-sum(D./p)
        invD <- exp(alpha+D.)
        L. <- diag(1,p)
        L.[lower.tri(L., diag=FALSE)] <- par[f11.1:f111]
        for (i in 1:k) {
            rss <- colSums(backsolve(L.,(tx-mu[,i]), upper.tri=FALSE)^2/invD)
            invl <- invl+w[i]*exp(-0.5*(p*(alpha+l2pi)+rss))
        }
    },
    "VEE" = {
        alpha <- par[f:f2]
        D. <- par[f2.1:f21]
        D. <- c(-sum(D.), D.)
        D. <- D.-sum(D./p)
        L. <- diag(1,p)
        L.[lower.tri(L., diag=FALSE)] <- par[f21.1:f211]
        for (i in 1:k) {
            rss <- colSums(backsolve(L., (tx-mu[,i]), upper.tri=FALSE)^2/exp(alpha[i]+D.))
            invl <- invl+w[i]*exp(-0.5*(p*(alpha[i]+l2pi)+rss))
        }
    },
    "EVV" = {
        alpha <- par[f]
        D. <- matrix(par[f1.1:f12],p-1,k)
        D. <- apply(D.,2, function(j) c(-sum(j), j))
        D. <- apply(D.,2, function(j) j-sum(j)/p)
        L.temp <- matrix(par[f12.1:f121],p*(p-1)/2,k)
        for (i in 1:k) {
            L. <- diag(1,p)
            L.[lower.tri(L., diag=FALSE)] <- L.temp[,i]
            rss <- colSums(backsolve(L., (tx-mu[,i]), upper.tri=FALSE)^2/exp(alpha+D.[,i]))
            invl <- invl+w[i]*exp(-0.5*(p*(alpha+l2pi)+rss))
        }
    },
    "VVV" = {
        alpha <- par[f:f2]
        D. <- matrix(par[f2.1:f22],p-1,k)
        D. <- apply(D.,2, function(j) c(-sum(j), j))
        D. <- apply(D.,2, function(j) j-sum(j)/p)
        invalpha <- exp(rep(alpha, each=p)+D.)
        L.temp <- matrix(par[f22.1:f221],p*(p-1)/2,k)
        L. <- diag(1,p)
        for (i in 1:k) {
            L.[lower.tri(L., diag=FALSE)] <- L.temp[,i]
            rss <- colSums(backsolve(L., (tx-mu[,i]), upper.tri=FALSE)^2/invalpha[,i])
            invl <- invl+w[i]*exp(-0.5*(p*(alpha[i]+l2pi)+rss))
        }
    },
    ## otherwise
    stop("invalid model:", model)
    )
    ## return  sum_{i=1}^n log( f(tx_i) ) :
    sum(log(invl))
}
sllnorMmix <- function(x, obj, trafo=c("clr1", "logit")) {
    stopifnot(is.character(model <- obj$model))
    trafo <- match.arg(trafo)
    llnorMmix(nMm2par(obj, model=model),
              tx = t(x), k = obj$k, 
              model=model, trafo=trafo)
}
## log-likelihood function relying on mvtnorm function
#
# par:   parameter vector as calculated by nMm2par
# x:     matrix of samples
# k:     number of cluster
# trafo: transformation of weights
# model: assumed model of the distribution
llmvtnorm <- function(par, x, k,
                      trafo=c("clr1", "logit"),
                      model=c("EII","VII","EEI","VEI","EVI",
                              "VVI","EEE","VEE","EVV","VVV")
              ) {
    stopifnot(is.matrix(x),
              length(k <- as.integer(k)) == 1, k >= 1)
    model <- match.arg(model)
    trafo <- match.arg(trafo)
    p <- ncol(x)
    nmm <- par2nMm(par, p, k, model=model, trafo=trafo)
    ## FIXME (speed!):  dmvnorm(*, sigma= S) will do a chol(S) for each component
    ## -----  *instead* we already have LDL' and  chol(S) = sqrt(D) L' !!
    ## another par2*() function should give L and D, or from that chol(Sagma), rather than Sigma !
    w <- nmm$w
    mu <- nmm$mu
    sig <- nmm$Sigma
    y <- 0
    for (i in 1:k) {
        y <- y + w[i]*mvtnorm::dmvnorm(x,mean=mu[,i],sigma=sig[,,i])
    }
    sum(log(y))
}
\end{Soutput}
\end{Schunk}

\clearpage
\section{Example Simulation Script}

\label{App:sims}
% TODO:
here e.g. 2init.R and write some remarks on it.

\begin{Schunk}
\begin{Soutput}
## Intent: analyse time as function of p,k,n
nmmdir <- normalizePath("~/BachelorArbeit/norMmix.Rcheck/")
savdir <- normalizePath("~/BachelorArbeit/Rscripts/2time")
stopifnot(dir.exists(nmmdir), dir.exists(savdir))
library(norMmix, lib.loc=nmmdir)
library(mclust)
## at n=500,p=2 can do about 250xfitnMm(x,1:10) in 24h
seeds <- 1:10
sizes <- c(500, 1000, 2000)
nmm <- list(MW214, MW34, MW51)
## => about 100 cases
# for naming purposes
nmnames <- c("MW214", "MW34", "MW51")
sizenames <- c("500", "1000", "2000")
files <- vector(mode="character")
for (nm in 1:3) {
    for (size in sizes) {
    set.seed(2019); x <- rnorMmix(size, nmm[[nm]])
        for (seed in seeds) {
            set.seed(2019+seed)
            r <- tryCatch(fitnMm(x, k=1:8,
                                 optREPORT=1e4, maxit=1e4),
                          error = identity)
            filename <- sprintf("%s_size=%0.4d_seed=%0.2d.rds",
                                nmnames[nm], size, seed)
            files <- append(files, filename)
            cat("===> saving to file:", filename, "\n")
            saveRDS(list(fit=r), file=file.path(savdir, filename))
        }
    }
}
fillis <- list()
for (i in seq_along(sizes)) {
    for (j in seq_along(nmnames)) {
        # for lack of AND matching, OR match everything else and invert
        ret <- grep(paste(sizenames[-i], nmnames[-j], sep="|"), 
                    files, value=TRUE, invert=TRUE)
        fillis[[paste0(sizenames[i], nmnames[j])]] <- ret
    }
}
epfl(fillis, savdir)
\end{Soutput}
\end{Schunk}


\chapter{Further Plots}


here further plots:

\section{Ch3}

dsfasdf

% outcommented \include{Appendix1}
% outcommented \chapter{2nd Appendix: More sophisticated R code listing} \label{appendix-more-R}

Chapter-wise listing of parts of R code, using
\begin{itemize}
\item \texttt{firstline=n1}
\item \texttt{lastline=n2}
\item \texttt{title=<text>}
\end{itemize}
e.g., for the first example below
\begin{verbatim}
\lstinputlisting[firstline=1,lastline=32,
                 title= \texttt{read\_irwls\_fn.R}]{../RCode/read_irwls_fn.R}
\end{verbatim}

% \section{Chapter 2} \label{app 2}

% \lstinputlisting[firstline=1,lastline=77,
% title=\texttt{analytic\_efficiency.R}]{../RCode/analytic_efficiency.R}
% %\lstinputlisting[firstline=,lastline=]{../RCode/???.R}

\bigskip% or even  \clearpage

%-----------------------------------------------------------------------------------------
\section{Chapter 5} \label{app 5}

% \lstinputlisting[firstline=1,lastline=71,
%                  title=\texttt{loss-fn\_rotated.R}]{../RCode/loss-fn_rotated.R}
%\lstinputlisting[firstline=1,lastline=32,
%                 title= \texttt{read\_irwls\_fn.R}]{../RCode/read_irwls_fn.R}

\medskip
                 
%\lstinputlisting[firstline=1,lastline=45,title=\texttt{plot.psi.R}]{../RCode/plot.psi.R}
%\lstinputlisting[firstline=,lastline=]{../RCode/???.R}
%\lstinputlisting[firstline=,lastline=]{../RCode/???.R}

% \clearpage
%-----------------------------------------------------------------------------------------
% \section{Chapter 7} \label{app 7}

% \lstinputlisting[firstline=1,lastline=35,
%                  title= \texttt{stat.test} from \texttt{lmrob2-fn.R}]{../RCode/lmrob2-fn.R}
% \lstinputlisting[firstline=41,lastline=194,
%                  title=\texttt{M.optimal.ms} from \texttt{lmrob2-fn.R}]{../RCode/lmrob2-fn.R}
%\lstinputlisting[firstline=,lastline=]{../RCode/???.R}
%-----------------------------------------------------------------------------------------

%%% Local Variables:
%%% mode: latex
%%% TeX-master: "MasterThesisSfS"
%%% End:


%%%%%%%%%%%%%%%%%%%%%%%%%%%%%%%%%%%%%%%%%%%%%%%%%% 
%%% Declaration of originality (Do not remove!)%%%
%%%%%%%%%%%%%%%%%%%%%%%%%%%%%%%%%%%%%%%%%%%%%%%%%%
%% Instructions:
%% -------------
%% fill in the empty document confirmation-originality.pdf electronically
%% print it out and sign it
%% scan it in again and save the scan in this directory with name
%% confirmation-originality-scan.pdf 
%%
%% General info on plagiarism:
%% https://www.ethz.ch/students/en/studies/performance-assessments/plagiarism.html 
\cleardoublepage
\includepdf[pages={-}, frame=true,scale=1]{confirmation-originality-scan.pdf}
\end{document}

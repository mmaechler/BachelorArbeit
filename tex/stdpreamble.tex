\documentclass[a4paper, 12pt]{report}
\parindent9pt


\relpenalty=9999
\binoppenalty=9999

\usepackage[margin=1.2in]{geometry}
\usepackage{amsmath}
\usepackage{amssymb}
\usepackage{tikz-cd}
\usepackage{longtable}
\usepackage{amsthm}
\usepackage{color}   %May be necessary if you want to color links
\usepackage{hyperref}
\hypersetup{
	colorlinks=true, %set true if you want colored links
	linktoc=all,     %set to all if you want both sections and subsections linked
	linkcolor=blue,  %choose some color if you want links to stand out
	citecolor=blue,
	urlcolor  = blue,
}
\usepackage[normalem]{ulem}
\usepackage{marginnote}
\usepackage{mathabx}
\usepackage{theoremref}

%need to find this package

\DeclareFontFamily{U}{mathc}{}
\DeclareFontShape{U}{mathc}{m}{it}%
{<->s*[1.03] mathc10}{}

\DeclareMathAlphabet{\mathscr}{U}{mathc}{m}{it}


\newcommand{\ca}[1]{\mathcal{#1}}
\newcommand{\caf}{\mathcal{F}}
\newcommand{\cag}{\mathcal{G}}
\newcommand{\cau}{\mathcal{U}}
\newcommand{\caa}{\mathcal{A}}
\newcommand{\oxmod}{$\mathcal{O}_X$-module }
\newcommand{\oxmods}{$\mathcal{O}_X$-modules }
\newcommand{\oymod}{$\mathcal{O}_Y$-module }
\newcommand{\oymods}{$\mathcal{O}_Y$-modules }
\newcommand{\ox}{\mathcal{O}_X}
\newcommand{\oy}{\mathcal{O}_Y}
\newcommand{\oxx}{\mathcal{O}_{X,x}}
\newcommand{\oyy}{\mathcal{O}_{Y,y}}
%\newcommand{\hhom}{\mathcal{H}om}
\newcommand{\hhom}{\mathscr{Hom}}
\newcommand{\bbp}{\mathbb{P}}
\newcommand{\bbz}{\mathbb{Z}}




\theoremstyle{plain}
%\newtheorem{prop}{Proposition}[chapter]
%\newtheorem{lem}[prop]{Lemma}
%\newtheorem{up}[prop]{Universal Property}
%\newtheorem{thmdef}[prop]{Definition/Theorem}
%\newtheorem{thm}[prop]{Theorem}

%\theoremstyle{definition}
%\newtheorem{defn}{Definition}[chapter]
%\newtheorem{note}[defn]{Note}
%\newtheorem{fact}[defn]{Fact}
%\newtheorem{ex}[defn]{Example}



\DeclareMathOperator{\mor}{Mor}

\DeclareMathOperator{\ob}{Ob}

\DeclareMathOperator{\id}{id}

\DeclareMathOperator{\spec}{Spec}

\DeclareMathOperator{\kom}{Kom}

\DeclareMathOperator{\Hom}{Hom}

\DeclareMathOperator{\R}{R}

\DeclareMathOperator{\Ll}{L}

\DeclareMathOperator{\K}{K}

\DeclareMathOperator{\tot}{tot}
